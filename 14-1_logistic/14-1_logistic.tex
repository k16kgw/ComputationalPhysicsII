\documentclass[a4paper, oneside]{jsarticle}

\usepackage{listings,jlisting} %日本語のコメントアウトをする場合jlistingが必要
%ここからソースコードの表示に関する設定
\lstset{
%  language={Python},
  basicstyle={\ttfamily},
  identifierstyle={\small},
  commentstyle={\smallitshape},
  keywordstyle={\small\bfseries},
  ndkeywordstyle={\small},
  stringstyle={\small\ttfamily},
  frame={tb},
  breaklines=true,
  columns=[l]{fullflexible},
  numbers=left,
  xrightmargin=0zw,
  xleftmargin=3zw,
  numberstyle={\scriptsize},
  stepnumber=1,
  numbersep=1zw,
  lineskip=-0.5ex
}
%ここまでソースコードの表示に関する設定

% 自作マクロ
%\input{mymacro.tex}

% 図
%\usepackage{tikz}
%\usetikzlibrary{positioning}

\begin{document}

\title{14-1\_logistic.pyの実行結果のまとめ}
\author{香川渓一郎}
\date{\today}

\maketitle

\setcounter{tocdepth}{1}
\tableofcontents

\section{14-1\_logistic.pyのコード}

\begin{lstlisting}[caption=14-1\_logistic.py, label=logistic]
  import matplotlib.pyplot as plt
  import japanize_matplotlib         """日本語表示に対応"""
  
  N = 100         # 世代数
  
  # 初期設定
  x = float(input("初期値x0を入力してください:"))
  mu = float(input("増加率μの値を入力してください:"))
  x0 = x      # 初期値を保存
  
  # グラフ描画用変数
  xlist = [0]
  ylist = [x]
  
  for i in range(1, N):
      x = mu * x * (1 - x)
      # 随時グラフ描画用変数に代入
      xlist.append(i)
      ylist.append(x)
  
  #plt.rcParams["font.family"] = "IPAGothic"
  
  # グラフの表示
  fig = plt.figure()                  # グラフの描画先の準備
  plt.title('μ=%1.3f' %mu)
  plt.plot(xlist, ylist)
  plt.xlabel('世代数')
  plt.ylabel('(無次元化された)個体数')
  fig.savefig("x0_%1.3f-mu_%1.3f.png" %(x0, mu))     # グラフを画像に保存
  plt.show()
\end{lstlisting}

\end{document}