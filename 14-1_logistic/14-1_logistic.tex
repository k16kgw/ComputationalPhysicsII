\documentclass[a4paper, oneside]{jsarticle}

\usepackage{listings,jlisting} %日本語のコメントアウトをする場合jlistingが必要
%ここからソースコードの表示に関する設定
\lstset{
%  language={Python},
  basicstyle={\ttfamily},
  identifierstyle={\small},
  commentstyle={\smallitshape},
  keywordstyle={\small\bfseries},
  ndkeywordstyle={\small},
  stringstyle={\small\ttfamily},
  frame={tb},
  breaklines=true,
  columns=[l]{fullflexible},
  numbers=left,
  xrightmargin=0zw,
  xleftmargin=3zw,
  numberstyle={\scriptsize},
  stepnumber=1,
  numbersep=1zw,
  lineskip=-0.5ex
}
%ここまでソースコードの表示に関する設定

% 自作マクロ
\input{../mymacro.tex}

\usepackage{here}

% 図
%\usepackage{tikz}
%\usetikzlibrary{positioning}

\begin{document}

\title{14章1〜6節(ロジスティック写像)の実行結果のまとめ}
\author{香川渓一郎}
\date{\today}

\maketitle

\setcounter{tocdepth}{1}
\tableofcontents

\section{14-1\_logistic.pyのコード}

Landau, et al. 『計算物理学II』朝倉書店.14章p.338(1--6), p.340(1--2)の演習を次のプログラム 14-1\_logistic.py で行う.

\begin{lstlisting}[caption=14-1\_logistic.py, label=logistic]
  # -*- coding: utf-8 -*-
  """
  14-1_logistic.pyプログラム
  Landau, et al. 『計算物理学II』朝倉書店.14章p.338の演習
  (無次元化された)ロジスティック方程式のシミュレーション
  使い方 14-1_logistic.py
  """

  import matplotlib.pyplot as plt
  import japanize_matplotlib          # 日本語表示に対応
  
  N = 100         # 世代数
  
  # 初期設定
  x = float(input("初期値x0を入力してください:"))
  mu = float(input("増加率μの値を入力してください:"))
  x0 = x      # 初期値を保存
  
  # グラフ描画用変数
  xlist = [0]
  ylist = [x]
  
  """logistic方程式の計算"""
  for i in range(1, N):
      x = mu * x * (1 - x)
      # 随時グラフ描画用変数に代入
      xlist.append(i)
      ylist.append(x)
  
  # グラフの表示
  fig = plt.figure()                  # グラフの描画先の準備
  plt.title('初期値x0=%1.4f, 増加率μ=%1.4f' %(x0, mu))
  plt.plot(xlist, ylist)
  plt.xlabel('世代数')
  plt.ylabel('(無次元化された)個体数')
  fig.savefig("14-1_logistic/x0_%1.4f-mu_%1.4f.png" %(x0, mu))     # グラフをフォルダに画像として保存
  plt.show()
\end{lstlisting}

\section{14-1\_logistic.pyの実行結果}
\subsection{初期値を$0.75$に固定して増加率を変化させる}

課題14.3-3: $\mu<0$のとき個体数は振動しながら減衰し,$\mu=0$のとき個体数は直ちに消滅する.

\begin{figure}[H]
  \begin{tabular}{c}
    \begin{minipage}{0.50\hsize}
      \centering
      \includegraphics[width=70mm]
        {x0_0.7500-mu_-1.0000.png}
        \caption{初期値$x_0=0.75$,増加率$\mu=-1$}
        \label{fig:0.7500_-1.0000}
    \end{minipage}
    \begin{minipage}{0.50\hsize}
      \centering
      \includegraphics[width=70mm]
        {x0_0.7500-mu_-0.7500.png}
        \caption{初期値$x_0=0.75$,増加率$\mu=-0.75$}
        \label{fig:0.7500_-0.7500}
    \end{minipage}
  \end{tabular}
\end{figure}
\begin{figure}[H]
  \begin{tabular}{c}
    \begin{minipage}{0.50\hsize}
      \centering
      \includegraphics[width=70mm]
        {x0_0.7500-mu_-0.5000.png}
        \caption{初期値$x_0=0.75$,増加率$\mu=-0.5$}
        \label{fig:0.7500_-0.5000}
    \end{minipage}
    \begin{minipage}{0.50\hsize}
      \centering
      \includegraphics[width=70mm]
        {x0_0.7500-mu_-0.2500.png}
        \caption{初期値$x_0=0.75$,増加率$\mu=-0.25$}
        \label{fig:0.7500_-0.2500}
    \end{minipage}
  \end{tabular}
\end{figure}
\begin{figure}[H]
  \begin{tabular}{c}
    \begin{minipage}{0.50\hsize}
      \centering
      \includegraphics[width=70mm]
        {x0_0.7500-mu_0.0000.png}
        \caption{初期値$x_0=0.75$,増加率$\mu=0$}
        \label{fig:0.7500_0.0000}
    \end{minipage}
    \begin{minipage}{0.50\hsize}
      \centering
    \end{minipage}
  \end{tabular}
\end{figure}

課題14.3-4: $\mu \ge 0$のときの個体数の変化.
\begin{figure}[H]
  \begin{tabular}{c}
    \begin{minipage}{0.50\hsize}
      \centering
      \includegraphics[width=70mm]
        {x0_0.7500-mu_0.0000.png}
        \caption{初期値$x_0=0.75$,増加率$\mu=0$}
        \label{fig:0.7500_0.0000}
    \end{minipage}
    \begin{minipage}{0.50\hsize}
      \centering
      \includegraphics[width=70mm]
        {x0_0.7500-mu_0.5000.png}
        \caption{初期値$x_0=0.75$,増加率$\mu=0.5$}
        \label{fig:0.7500_0.5000}
    \end{minipage}
  \end{tabular}
\end{figure}
\begin{figure}[H]
  \begin{tabular}{c}
    \begin{minipage}{0.50\hsize}
      \centering
      \includegraphics[width=70mm]
        {x0_0.7500-mu_1.0000.png}
        \caption{初期値$x_0=0.75$,増加率$\mu=1$}
        \label{fig:0.7500_1.0000}
    \end{minipage}
    \begin{minipage}{0.50\hsize}
      \centering
      \includegraphics[width=70mm]
        {x0_0.7500-mu_1.3000.png}
        \caption{初期値$x_0=0.75$,増加率$\mu=1.3$}
        \label{fig:0.7500_1.3000}
    \end{minipage}
  \end{tabular}
\end{figure}
\begin{figure}[H]
  \begin{tabular}{c}
    \begin{minipage}{0.50\hsize}
      \centering
      \includegraphics[width=70mm]
        {x0_0.7500-mu_1.3300.png}
        \caption{初期値$x_0=0.75$,増加率$\mu=1.33$}
        \label{fig:0.7500_1.3300}
    \end{minipage}
    \begin{minipage}{0.50\hsize}
      \centering
      \includegraphics[width=70mm]
        {x0_0.7500-mu_1.3500.png}
        \caption{初期値$x_0=0.75$,増加率$\mu=1.35$}
        \label{fig:0.7500_1.3500}
    \end{minipage}
  \end{tabular}
\end{figure}
\begin{figure}[H]
  \begin{tabular}{c}
    \begin{minipage}{0.50\hsize}
      \centering
      \includegraphics[width=70mm]
        {x0_0.7500-mu_1.5000.png}
        \caption{初期値$x_0=0.75$,増加率$\mu=1.5$}
        \label{fig:0.7500_1.5000}
    \end{minipage}
    \begin{minipage}{0.50\hsize}
      \centering
      \includegraphics[width=70mm]
        {x0_0.7500-mu_2.0000.png}
        \caption{初期値$x_0=0.75$,増加率$\mu=2$}
        \label{fig:0.7500_2.0000}
    \end{minipage}
  \end{tabular}
\end{figure}
\begin{figure}[H]
  \begin{tabular}{c}
    \begin{minipage}{0.50\hsize}
      \centering
      \includegraphics[width=70mm]
        {x0_0.7500-mu_2.8000.png}
        \caption{初期値$x_0=0.75$,増加率$\mu=2.8$}
        \label{fig:0.750_2.800}
    \end{minipage}
    \begin{minipage}{0.50\hsize}
      \centering
      \includegraphics[width=70mm]
        {x0_0.7500-mu_2.9000.png}
        \caption{初期値$x_0=0.75$,増加率$\mu=2.9$}
        \label{fig:0.750_2.900}
    \end{minipage}
  \end{tabular}
\end{figure}
\begin{figure}[H]
  \begin{tabular}{c}
    \begin{minipage}{0.50\hsize}
      \centering
      \includegraphics[width=70mm]
        {x0_0.7500-mu_3.0000.png}
        \caption{初期値$x_0=0.75$,増加率$\mu=3$}
        \label{fig:0.750_3.000}
    \end{minipage}
    \begin{minipage}{0.50\hsize}
      \centering
      \includegraphics[width=70mm]
        {x0_0.7500-mu_3.5000.png}
        \caption{初期値$x_0=0.75$,増加率$\mu=3.5$}
        \label{fig:0.750_3.500}
    \end{minipage}
  \end{tabular}
\end{figure}
\begin{figure}[H]
  \begin{tabular}{c}
    \begin{minipage}{0.50\hsize}
      \centering
      \includegraphics[width=70mm]
        {x0_0.7500-mu_3.8000.png}
        \caption{初期値$x_0=0.75$,増加率$\mu=3.8$}
        \label{fig:0.7500_3.8000}
    \end{minipage}
    \begin{minipage}{0.50\hsize}
      \centering
      \includegraphics[width=70mm]
        {x0_0.7500-mu_4.0000.png}
        \caption{初期値$x_0=0.75$,増加率$\mu=4$}
        \label{fig:0.7500_4.0000}
    \end{minipage}
  \end{tabular}
\end{figure}
\begin{figure}[H]
  \begin{tabular}{c}
    \begin{minipage}{0.50\hsize}
      \centering
      \includegraphics[width=70mm]
        {x0_0.7500-mu_4.1000.png}
        \caption{初期値$x_0=0.75$,増加率$\mu=4.1$}
        \label{fig:0.7500_3.8000}
    \end{minipage}
  \end{tabular}
\end{figure}

\newpage
\subsection{増加率を固定して初期値を変化させる}

課題14.3-6: 増加率$\mu$を一定として初期値$x_0$を変化させる.

$\mu=1$で固定した場合
\begin{figure}[H]
  \begin{tabular}{c}
    \begin{minipage}{0.50\hsize}
      \centering
      \includegraphics[width=70mm]
        {x0_0.1000-mu_1.0000.png}
        \caption{初期値$x_0=0.1$,増加率$\mu=1$}
        \label{fig:0.1000_1.0000}
    \end{minipage}
    \begin{minipage}{0.50\hsize}
      \centering
      \includegraphics[width=70mm]
        {x0_0.5000-mu_1.0000.png}
        \caption{初期値$x_0=0.5$,増加率$\mu=1$}
        \label{fig:0.5000_1.0000}
    \end{minipage}
  \end{tabular}
\end{figure}
\begin{figure}[H]
  \begin{tabular}{c}
    \begin{minipage}{0.50\hsize}
      \centering
      \includegraphics[width=70mm]
        {x0_0.7500-mu_1.0000.png}
        \caption{初期値$x_0=0.75$,増加率$\mu=1$}
        \label{fig:0.7500_1.0000-2}
    \end{minipage}
    \begin{minipage}{0.50\hsize}
      \centering
      \includegraphics[width=70mm]
        {x0_0.9000-mu_1.0000.png}
        \caption{初期値$x_0=0.9$,増加率$\mu=1$}
        \label{fig:0.9000_1.0000}
    \end{minipage}
  \end{tabular}
\end{figure}

$\mu=2.9$で固定した場合

$x_* \ne 0$を不動点とするとき,
\begin{equation}
  x_* = \mu x_* (1-x_*)
\end{equation}
より
\begin{equation}
  x_* = 1 - \frac{1}{\mu}.
\end{equation}
$\mu=2.9$とすると$x_*=0.65517\ldots$.
\begin{figure}[H]
  \begin{tabular}{c}
    \begin{minipage}{0.50\hsize}
      \centering
      \includegraphics[width=70mm]
        {x0_0.1000-mu_2.9000.png}
        \caption{初期値$x_0=0.1$,増加率$\mu=2.9$}
        \label{fig:0.1000_2.9000}
    \end{minipage}
    \begin{minipage}{0.50\hsize}
      \centering
      \includegraphics[width=70mm]
        {x0_0.2500-mu_2.9000.png}
        \caption{初期値$x_0=0.25$,増加率$\mu=2.9$}
        \label{fig:0.2500_2.9000}
    \end{minipage}
  \end{tabular}
\end{figure}
\begin{figure}[H]
  \begin{tabular}{c}
    \begin{minipage}{0.50\hsize}
      \centering
      \includegraphics[width=70mm]
        {x0_0.5000-mu_2.9000.png}
        \caption{初期値$x_0=0.5$,増加率$\mu=2.9$}
        \label{fig:0.5000_2.9000}
    \end{minipage}
    \begin{minipage}{0.50\hsize}
      \centering
      \includegraphics[width=70mm]
        {x0_0.6500-mu_2.9000.png}
        \caption{初期値$x_0=0.65$,増加率$\mu=2.9$}
        \label{fig:0.6500_2.9000}
    \end{minipage}
  \end{tabular}
\end{figure}
\begin{figure}[H]
  \begin{tabular}{c}
    \begin{minipage}{0.50\hsize}
      \centering
      \includegraphics[width=70mm]
        {x0_0.6540-mu_2.9000.png}
        \caption{初期値$x_0=0.654$,増加率$\mu=2.9$}
        \label{fig:0.6540_2.9000}
    \end{minipage}
    \begin{minipage}{0.50\hsize}
      \centering
      \includegraphics[width=70mm]
        {x0_0.6552-mu_2.9000.png}
        \caption{初期値$x_0=0.6552$,増加率$\mu=2.9$}
        \label{fig:0.6552_2.9000}
    \end{minipage}
  \end{tabular}
\end{figure}
\begin{figure}[H]
  \begin{tabular}{c}
    \begin{minipage}{0.50\hsize}
      \centering
      \includegraphics[width=70mm]
        {x0_0.7500-mu_2.9000.png}
        \caption{初期値$x_0=0.75$,増加率$\mu=2.9$}
        \label{fig:0.7500_2.9000-2}
    \end{minipage}
    \begin{minipage}{0.50\hsize}
      \centering
      \includegraphics[width=70mm]
        {x0_0.9000-mu_2.9000.png}
        \caption{初期値$x_0=0.9$,増加率$\mu=2.9$}
        \label{fig:0.9000_2.9000}
    \end{minipage}
  \end{tabular}
\end{figure}

\newpage
$\mu=3.3$で固定した場合
\begin{figure}[H]
  \begin{tabular}{c}
    \begin{minipage}{0.50\hsize}
      \centering
      \includegraphics[width=70mm]
        {x0_0.1000-mu_3.3000.png}
        \caption{初期値$x_0=0.1$,増加率$\mu=3.3$}
        \label{fig:0.1000_3.3000}
    \end{minipage}
    \begin{minipage}{0.50\hsize}
      \centering
      \includegraphics[width=70mm]
        {x0_0.2500-mu_3.3000.png}
        \caption{初期値$x_0=0.25$,増加率$\mu=3.3$}
        \label{fig:0.2500_3.3000}
    \end{minipage}
  \end{tabular}
\end{figure}
\begin{figure}[H]
  \begin{tabular}{c}
    \begin{minipage}{0.50\hsize}
      \centering
      \includegraphics[width=70mm]
        {x0_0.5000-mu_3.3000.png}
        \caption{初期値$x_0=0.5$,増加率$\mu=3.3$}
        \label{fig:0.5000_3.3000}
    \end{minipage}
    \begin{minipage}{0.50\hsize}
      \centering
      \includegraphics[width=70mm]
        {x0_0.7000-mu_3.3000.png}
        \caption{初期値$x_0=0.7$,増加率$\mu=3.3$}
        \label{fig:0.7000_3.3000}
    \end{minipage}
  \end{tabular}
\end{figure}
\begin{figure}[H]
  \begin{tabular}{c}
    \begin{minipage}{0.50\hsize}
      \centering
      \includegraphics[width=70mm]
        {x0_0.9000-mu_3.3000.png}
        \caption{初期値$x_0=0.9$,増加率$\mu=3.3$}
        \label{fig:0.9000_3.3000}
    \end{minipage}
  \end{tabular}
\end{figure}

\newpage
$\mu=3.5$で固定した場合
\begin{figure}[H]
  \begin{tabular}{c}
    \begin{minipage}{0.50\hsize}
      \centering
      \includegraphics[width=70mm]
        {x0_0.1000-mu_3.5000.png}
        \caption{初期値$x_0=0.1$,増加率$\mu=3.5$}
        \label{fig:0.1000_3.5000}
    \end{minipage}
    \begin{minipage}{0.50\hsize}
      \centering
      \includegraphics[width=70mm]
        {x0_0.2500-mu_3.5000.png}
        \caption{初期値$x_0=0.25$,増加率$\mu=3.5$}
        \label{fig:0.2500_3.5000}
    \end{minipage}
  \end{tabular}
\end{figure}
\begin{figure}[H]
  \begin{tabular}{c}
    \begin{minipage}{0.50\hsize}
      \centering
      \includegraphics[width=70mm]
        {x0_0.5000-mu_3.5000.png}
        \caption{初期値$x_0=0.5$,増加率$\mu=3.5$}
        \label{fig:0.5000_3.5000}
    \end{minipage}
    \begin{minipage}{0.50\hsize}
      \centering
      \includegraphics[width=70mm]
        {x0_0.7500-mu_3.5000.png}
        \caption{初期値$x_0=0.75$,増加率$\mu=3.5$}
        \label{fig:0.7500_3.5000-2}
    \end{minipage}
  \end{tabular}
\end{figure}
\begin{figure}[H]
  \begin{tabular}{c}
    \begin{minipage}{0.50\hsize}
      \centering
      \includegraphics[width=70mm]
        {x0_0.8000-mu_3.5000.png}
        \caption{初期値$x_0=0.8$,増加率$\mu=3.5$}
        \label{fig:0.8000_3.5000}
    \end{minipage}
    \begin{minipage}{0.50\hsize}
      \centering
      \includegraphics[width=70mm]
        {x0_0.9000-mu_3.5000.png}
        \caption{初期値$x_0=0.9$,増加率$\mu=3.5$}
        \label{fig:0.9000_3.5000}
    \end{minipage}
  \end{tabular}
\end{figure}

\newpage
$\mu=3.8$で固定した場合
\begin{figure}[H]
  \begin{tabular}{c}
    \begin{minipage}{0.50\hsize}
      \centering
      \includegraphics[width=70mm]
        {x0_0.1000-mu_3.8000.png}
        \caption{初期値$x_0=0.1$,増加率$\mu=3.8$}
        \label{fig:0.1000_3.8000}
    \end{minipage}
    \begin{minipage}{0.50\hsize}
      \centering
      \includegraphics[width=70mm]
        {x0_0.2500-mu_3.8000.png}
        \caption{初期値$x_0=0.25$,増加率$\mu=3.8$}
        \label{fig:0.2500_3.8000}
    \end{minipage}
  \end{tabular}
\end{figure}
\begin{figure}[H]
  \begin{tabular}{c}
    \begin{minipage}{0.50\hsize}
      \centering
      \includegraphics[width=70mm]
        {x0_0.5000-mu_3.8000.png}
        \caption{初期値$x_0=0.5$,増加率$\mu=3.8$}
        \label{fig:0.5000_3.8000}
    \end{minipage}
    \begin{minipage}{0.50\hsize}
      \centering
      \includegraphics[width=70mm]
        {x0_0.7500-mu_3.8000.png}
        \caption{初期値$x_0=0.75$,増加率$\mu=3.8$}
        \label{fig:0.7500_3.8000-2}
    \end{minipage}
  \end{tabular}
\end{figure}
\begin{figure}[H]
  \begin{tabular}{c}
    \begin{minipage}{0.50\hsize}
      \centering
      \includegraphics[width=70mm]
        {x0_0.9000-mu_3.8000.png}
        \caption{初期値$x_0=0.9$,増加率$\mu=3.8$}
        \label{fig:0.9000_3.8000}
    \end{minipage}
  \end{tabular}
\end{figure}

\newpage
\subsection{ファイゲンバウムによる観察の確認}

ここでは初期値を$x_0=0.75$に固定する.
\begin{figure}[H]
  \begin{tabular}{c}
    \begin{minipage}{0.50\hsize}
      \centering
      \includegraphics[width=70mm]
        {x0_0.7500-mu_-0.5000.png}
        \caption{増加率$\mu=-0.5$のとき個体数は定常解のゼロに漸近し,時間が経つとゼロで安定する}
        \label{fig:0.7500_-0.5000-2}
    \end{minipage}
    \begin{minipage}{0.50\hsize}
      \centering
      \includegraphics[width=70mm]
        {x0_0.7500-mu_0.4000.png}
        \caption{増加率$\mu=0.4$のとき個体数は定常解のゼロに漸近し,時間が経つとゼロで安定する}
        \label{fig:0.7500_0.4000}
    \end{minipage}
  \end{tabular}
\end{figure}
\begin{figure}[H]
  \begin{tabular}{c}
    \begin{minipage}{0.50\hsize}
      \centering
      \includegraphics[width=70mm]
        {x0_0.7500-mu_1.5000.png}
        \caption{増加率$\mu=1.5$のとき個体数は非自明な定常解に漸近し,時間が経つとその定常解で安定する}
        \label{fig:0.7500_1.5000-2}
    \end{minipage}
    \begin{minipage}{0.50\hsize}
      \centering
      \includegraphics[width=70mm]
        {x0_0.7500-mu_2.4000.png}
        \caption{増加率$\mu=2.4$のとき個体数は非自明な定常解に振動しながら漸近し,時間が経つとその定常解で安定する}
        \label{fig:0.7500_2.4000}
    \end{minipage}    
  \end{tabular}
\end{figure}
\begin{figure}[H]
  \begin{tabular}{c}
    \begin{minipage}{0.50\hsize}
      \centering
      \includegraphics[width=70mm]
        {x0_0.7500-mu_3.2000.png}
        \caption{増加率$\mu=3.2$のとき時間が経つと二重周期解に安定する}
        \label{fig:0.7500_3.2000}
    \end{minipage}
    \begin{minipage}{0.50\hsize}
      \centering
      \includegraphics[width=70mm]
        {x0_0.7500-mu_3.5000.png}
        \caption{増加率$\mu=3.5$のとき時間が経つと四重周期解に安定する}
        \label{fig:0.7500_3.5000-2}
    \end{minipage}    
  \end{tabular}
\end{figure}
\begin{figure}[H]
  \begin{tabular}{c}
    \begin{minipage}{0.50\hsize}
      \centering
      \includegraphics[width=70mm]
        {x0_0.7500-mu_3.6000.png}
        \caption{増加率$\mu=3.6$のときカオス的な振る舞いをする}
        \label{fig:0.7500_3.6000}
    \end{minipage}
    \begin{minipage}{0.50\hsize}
      \centering
      \includegraphics[width=70mm]
        {x0_0.7500-mu_3.8264.png}
        \caption{増加率$\mu=3.8264$のとき安定に見える期間とそれが崩れたのち再び安定に見える状態となる間欠状態が見られる}
        \label{fig:0.7500_3.8264}
    \end{minipage}    
  \end{tabular}
\end{figure}
\begin{figure}[H]
  \begin{tabular}{c}
    \begin{minipage}{0.50\hsize}
      \centering
      \includegraphics[width=70mm]
        {x0_0.7500-mu_3.8274.png}
        \caption{増加率$\mu=3.8274$のとき間欠状態が見られる}
        \label{fig:0.7500_3.8274}
    \end{minipage}
    \begin{minipage}{0.50\hsize}
      \centering
      \includegraphics[width=70mm]
        {x0_0.7500-mu_3.8284-N_300.png}
        \caption{増加率$\mu=3.8284$のとき間欠状態が見られる(但し世代数の最大値を300回に増やす)}
        \label{fig:0.7500_3.8284}
    \end{minipage}    
  \end{tabular}
\end{figure}
\begin{figure}[H]
  \begin{tabular}{c}
    \begin{minipage}{0.50\hsize}
      \centering
      \includegraphics[width=70mm]
        {x0_0.7500-mu_3.8304.png}
        \caption{増加率$\mu=3.8304$のとき二重周期解で安定となる}
        \label{fig:0.7500_3.8304}
    \end{minipage}
    \begin{minipage}{0.50\hsize}
      \centering
      \includegraphics[width=70mm]
        {x0_0.7500-mu_4.0000.png}
        \caption{増加率$\mu=4$のとき初期値$0.75$が安定である}
        \label{fig:0.7500_4.0000}
    \end{minipage}    
  \end{tabular}
\end{figure}

\section{14-4\_chaos.pyのコード}

Landau, et al. 『計算物理学II』朝倉書店.14章p.340(3)の演習を次のコード 14-4\_chaos.py で行う.

\begin{lstlisting}[caption=14-4\_chaos.py, label=chaos]
  # -*- coding: utf-8 -*-
  """
  14-1_chaos.pyプログラム
  Landau, et al. 『計算物理学II』朝倉書店.14章p.340の演習
  (無次元化された)ロジスティック方程式のシミュレーション
  解のカオス的挙動を調べる
  使い方 14-1_chaos.py
  """
  
  import numpy as np 
  import matplotlib.pyplot as plt
  import japanize_matplotlib          # 日本語表示に対応
  
  N = 100         # 世代数
  
  # 初期設定
  x0 = 0.75
  mu = 3.95
  eps = 2e-14
  
  x0p = x0 * (1 + eps)
  mup = mu * (1 - eps)
  
  # 初期値の保存
  x1 = x0
  x2 = x0p
  x3 = x0
  
  # グラフ描画用変数
  xlist = [0]
  ylist1 = [x1]
  ylist2 = [x2]
  ylist3 = [x3]
  
  """logistic方程式の計算"""
  for i in range(1, N):
      x1 = mu * x1 * (1 - x1)
      x2 = mu * x2 * (1 - x2)
      x3 = mup * x3 * (1 - x3)
      # 随時グラフ描画用変数に代入
      xlist.append(i)
      ylist1.append(x1)
      ylist2.append(x2)
      ylist3.append(x3)
  
  """個別のグラフを並べて表示"""
  # グラフ描画先の準備
  fig1 = plt.figure()
  # グラフを表示する領域の追加
  ax1 = plt.subplot(3, 1, 1)          
  ax2 = plt.subplot(3, 1, 2)
  ax3 = plt.subplot(3, 1, 3)
  # 各グラフタイトル
  ax1.set_title('$x_0, \mu$')
  ax2.set_title('$x_0(1+\epsilon), \mu$')
  ax3.set_title('$x_0, \mu(1-\epsilon)$')
  # 各グラフの設定
  c1, c2, c3 = "blue", "red", "green"                     # プロットの色
  l1, l2, l3 = "$x_0, \mu$", "$x_0(1+\epsilon), \mu$", "$x_0, \mu(1-\epsilon)$" # ラベル
  ax1.plot(xlist, ylist1, color=c1, label=l1)
  ax2.plot(xlist, ylist2, color=c2, label=l2)
  ax3.plot(xlist, ylist3, color=c3, label=l3)
  # 軸名
  plt.xlabel('世代数')
  plt.ylabel('(無次元化された)個体数')
  
  fig1.suptitle('グラフを並べて表示', x=0.5, y=0.995)
  fig1.tight_layout()       # レイアウト設定
  fig1.savefig("14-1_logistic/chaos-parallel.png")     # グラフをフォルダに画像として保存
  plt.show()      # グラフの表示
  
  """グラフを重ねて表示"""
  fig2 = plt.figure()     # グラフ描画先の準備
  ax = plt.subplot()      # グラフを表示する領域の追加
  
  ax.set_title('グラフを重ねて表示')
  ax.set_xlabel('世代数')
  ax.set_ylabel('(無次元化された)個体数')
  
  ax.plot(xlist, ylist1, color=c1, label=l1)
  ax.plot(xlist, ylist2, color=c2, label=l2)
  ax.plot(xlist, ylist3, color=c3, label=l3)
  ax.legend(loc=0)    # 凡例
  fig2.tight_layout() # レイアウトの設定
  fig2.savefig('14-1_logistic/chaos-overlap.png')   # 画像の保存
  plt.show()  
\end{lstlisting}

\section{14-4\_chaos.pyの実行結果}

\begin{figure}[H]
  \centering
  \includegraphics[width=150mm]{chaos-parallel.png}
    \caption{ロジスティック写像のカオス的挙動-1}
    \label{fig:chaos-parallel}
\end{figure}
\begin{figure}[H]
  \centering
  \includegraphics[width=150mm]{chaos-overlap.png}
    \caption{ロジスティック写像のカオス的挙動-1}
    \label{fig:chaos-overlap}
\end{figure}

\newpage
\section{14-5\_branch.pyのコード}

Landau, et al. 『計算物理学II』朝倉書店.14章p.343(1--9)の演習を次のプログラム 14-8\_branch.py で行う.

\begin{lstlisting}[caption=14-8\_branch.py, label=branch]
  # -*- coding: utf-8 -*-
  """
  14-8_branch.pyプログラム
  Landau, et al. 『計算物理学II』朝倉書店.14章p.343の演習
  ロジスティック写像の分岐図を描画するシミュレーション
  使い方 14-8_branch.py
  """

  import numpy as np 
  import matplotlib.pyplot as plt 
  import japanize_matplotlib
  
  N = 300     # N+200:経過させる世代数
  
  # グラフ描画用変数
  xlist = []
  ylist = []
  
  print('start')
  for mu in np.arange(1, 4, 0.003):
      #print(mu)
      for x0 in np.arange(0.002, 1, 0.002):
          x = x0
          # 始めに200世代経過させる
          for i in range(1, 200):
              x = mu * x * (1-x)
          for i in range(1, N):
              x_before = x
              x = mu * x * (1-x)
              if x_before == x:
                  break
          ylist.append(float(format(x, '.4f')))
          #ylist.append(x)
          xlist.append(mu)
  
  print('グラフ描画開始')
  fig = plt.figure(figsize=(15, 15))
  plt.title('分岐図')
  plt.scatter(xlist, ylist, s=0.3)
  plt.xlabel('$\mu$')
  plt.ylabel('$x_*$')
  fig.savefig('14-1_logistic/branch.png', dpi=300)   # 画像の保存
  plt.show()
  
  print('グラフ描画終了')  
\end{lstlisting}

\newpage
\section{14-5\_branch.pyの実行結果}

\begin{figure}[H]
  \centering
  \includegraphics[width=150mm]{branch_1-4.png}
    \caption{$1<\mu<4$でのロジスティック写像の分岐図}
    \label{fig:branch1}
\end{figure}

\begin{figure}[H]
  \centering
  \includegraphics[width=150mm]{branch_3-4.png}
    \caption{$3<\mu<4$でのロジスティック写像の分岐図}
    \label{fig:branch2}
\end{figure}

\end{document}